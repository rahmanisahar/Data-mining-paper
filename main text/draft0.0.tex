\documentclass[useAMS,usenatbib]{mn2e}
\usepackage{amsmath}
\usepackage{hyperref}
\usepackage{graphicx}
\usepackage{natbib}
\usepackage{times}
\usepackage{float}
%\usepackage{caption}
\usepackage{subcaption}
\usepackage{multirow}
\usepackage{color,soul}
\usepackage{import}
\usepackage[T1]{fontenc}
\usepackage{ae,aecompl}
\usepackage{amssymb}	% Extra maths symbols
\usepackage{multicol}        % Multi-column entries in tables
\usepackage{bm}		% Bold maths symbols, including upright Greek
\usepackage{pdflscape}	% Landscape pages
%\usepackage{booktabs,fixltx2e}
\usepackage[flushleft]{threeparttable}
\usepackage{mwe}    % loads »blindtext« and »graphicx«
\usepackage{subfig}

\newcommand \kpc        {\,{\rm kpc}}
\newcommand \sigmagas    {$\Sigma_{\rm \bld {gas }} $\ }
\newcommand \sigmatotalgas {$\Sigma_{\rm \bld {total\, gas }} $\ }
\newcommand \eqsigmagas    {\Sigma_{\rm \bld {gas }}}
\newcommand \sigmasfr     {$\Sigma_{\rm \bld {SFR }} $\ }
\newcommand \eqsigmasfr     {\Sigma_{\rm \bld {SFR }}}
\newcommand \sigmastar    {$\Sigma_{\rm \bld {star }} $\ }
\newcommand \eqsigmastar    {\Sigma_{\rm \bld {star }}}
\newcommand \halpha    {H$\alpha $}
\newcommand \um    {$\mu$m}
%\newcommand \mice {$\mu$m}
\newcommand \nprime {N$^\prime$}
\newcommand \boldit {\textbf{\textit{}}}
\newcommand \eqnprime {N^\prime}
\newcommand \Spitzer {{\it Spitzer }}
\newcommand \GALEX {GALEX }
\newcommand \Herschel {{\it Herschel }}
\newcommand \sii {[S~{\textsc II}]} 
\newcommand \oiii {[O~{\textsc III}]} 
\newcommand \hi {H~{\textsc I}\ }

\newcommand \aaj {A\&A}
\newcommand \aarv {A\&ARv}%: Astronomy and Astrophysics Review (the)
\newcommand \aas{A\&AS}%: Astronomy and Astrophysics Supplement Series
\newcommand \afz {Afz}%: Astrofizika
\newcommand \aj {AJ}%: Astronomical Journal (the)
\newcommand \apss {Ap\&SS}%: Astrophysics and Space Science
\newcommand \apj {ApJ}
\newcommand \apjs {ApJS}%: Astrophysical Journal Supplement Series (the)
\newcommand \araa {ARA\&A} %: Annual Review of Astronomy and Astrophysics
\newcommand \asp {ASP Conf. Ser.}%: Astronomy Society of the Pacific Conference Series
\newcommand \azh {Azh}%: Astronomicheskij Zhurnal
\newcommand \baas {BAAS}%: Bulletin of the American Astronomical Society
\newcommand \mem {Mem. RAS}%: Memoirs of the Royal Astronomical Society
\newcommand \mnassa {MNASSA}%: Monthly Notes of the Astronomical Society of Southern Africa
\newcommand \mnras {MNRAS} %: Monthly Notices of the Royal Astronomical Society
\newcommand \pasj {PASJ}%: Publications of the Astronomical Society of Japan
\newcommand \pasp {PASP}%: Publications of the Astronomical Society of the Pacific
\newcommand \qjras {QJRAS}%: Quarterly Journal of the Royal Astronomical Society
\newcommand \mex {Rev. Mex. Astron. Astrofis.}%: Revista Mexicana de Astronomia y Astrofisica
\newcommand \sva {SvA}%: Soviet Astronomy
\newcommand \aap {APP} %:American Academy of Pediatrics
\newcommand \apjl {ApJL} %:The Astrophysical Journal Letters


\begin{document}
\title{Data Mining in Nearby Galaxies}
\author[S. Rahmani, et. al.]{S.~Rahmani$^{1, 2}$\thanks{E-mail:
srahma49@uwo.ca}, H.~Teimoorinia$^{3}$, E.~Peeters$^{1, 2, 4}$, P.~Barmby$^{1, 2}$\\
$^{1}$Department of Physics $\&$ Astronomy, Western University, London, ON N6A 3K7, Canada\\
$^{2}$Center for Planetary Science \& Exploration, Western University, London, ON N6A 3K7, Canada\\
$^{3}$Department of Physics $\&$ Astronomy, University of Victoria, Finnerty Road, Victoria, British Columbia, V8P 1A1, Canada\\
$^{4}$SETI Institute, 189 Bernardo Avenue, Suite 100, Mountain View, CA 94043, USA}
\maketitle

%----------------------------------------------------------------------------------------
%----------------------------------------------------------------------------------------
%----------------------------------------------------------------------------------------
%abstract
%----------------------------------------------------------------------------------------
%----------------------------------------------------------------------------------------
%----------------------------------------------------------------------------------------

\begin{abstract} 
Galaxies are very complex systems; therefore, a complete understanding of them cannot be achieved only from studying linear or logarithmic correlations between their properties and various waveband data.
In the past few decades, a number of statistical methods have been developed and advanced to study and visualize complex big data.
We used one of these methods to study properties of nearby galaxies and create a more clear picture of them.
The vast availability of data for nearby galaxies makes them suitable targets for exploratory data analysis. 
Spatially resolved maps provide us with a unique view of the inside of galaxies. %and help better understand their properties.
In this project, we utilized the Kohonen Self Organizing Map (SOM) method to study data from M31 and M101. %SOM is an unsupervised neural network for mapping and visualizing a complex and nonlinear high dimension data while preserving topological features of the original data. 
%We studied 10 regions in M31 and 8 regions in M101. 
%These regions were chosen based on the availability of mid-infrared spectroscopy data.
%For each region in M31, we had 46 data obtained through photometry, spectroscopy, and derived quantities (i.e., star formation rate, stellar mass, gas mass, etc.). 
%Using results from a correlation coefficient matrix, we reduced the dimension of this dataset and then trained the SOM using the new subset of data. 
We created SOMs of various sizes and used them to extract information about the galaxies.
Using smaller-sized SOMs the data were clustered in 2 major groups; and for each group, we found correlations that could not have otherwise seen without clustering. 
In the maps with larger sizes, we created networks to illustrate the relative relations of the regions with one another. 
We then applied the SOMs that were generated from the M31 data to from M101. 
We found regions with similar properties in both galaxies placed in close regions in the SOMs.  
These results confirm that the generated SOMs can separate regions based on their physical properties, and can be used to make predictions for other regions in nearby galaxies or other targets. 
\end{abstract}
\begin{keywords} 
galaxies: individual: M31, galaxies: spiral, galaxies: star formation, galaxies: stellar content, galaxies: ISM, stars: formation, ISM: clouds, methods: observational, methods: statistical, data mining, methods:data analysis, techniques: image processing 
\end{keywords}
%----------------------------------------------------------------------------------------
%----------------------------------------------------------------------------------------
%----------------------------------------------------------------------------------------
%Intro
%----------------------------------------------------------------------------------------
%----------------------------------------------------------------------------------------
%----------------------------------------------------------------------------------------
\import{../sections/}{intro0.0.tex}
%----------------------------------------------------------------------------------------
%----------------------------------------------------------------------------------------
%----------------------------------------------------------------------------------------
%DATA
%----------------------------------------------------------------------------------------
%----------------------------------------------------------------------------------------
%----------------------------------------------------------------------------------------

\import{../sections/}{data0.0.tex}

%----------------------------------------------------------------------------------------
%----------------------------------------------------------------------------------------
%----------------------------------------------------------------------------------------
%Method
%----------------------------------------------------------------------------------------
%----------------------------------------------------------------------------------------
%----------------------------------------------------------------------------------------

\import{../sections/}{method0.0.tex}

%----------------------------------------------------------------------------------------
%----------------------------------------------------------------------------------------
%----------------------------------------------------------------------------------------
%Results
%----------------------------------------------------------------------------------------
%----------------------------------------------------------------------------------------
%----------------------------------------------------------------------------------------
%\import{../sections/}{results0.0.tex}
\import{../sections/}{R10.0.tex}
\import{../sections/}{R20.0.tex}
%----------------------------------------------------------------------------------------
%----------------------------------------------------------------------------------------
%----------------------------------------------------------------------------------------
%Discussion
%----------------------------------------------------------------------------------------
%----------------------------------------------------------------------------------------
%----------------------------------------------------------------------------------------

%\import{../sections/}{caveat.tex}

%----------------------------------------------------------------------------------------
%----------------------------------------------------------------------------------------
%----------------------------------------------------------------------------------------
%Summery
%----------------------------------------------------------------------------------------
%----------------------------------------------------------------------------------------
%----------------------------------------------------------------------------------------
\section{SUMMARY}
\label{sec: summary}

We present results of studying spatially resolved regions in nearby galaxies using 
The Self-organizing maps both cluster data and show relative distance between each cluster. 
Therefore, we utilized it to find hidden subgroups in the data as well as finding relation between those groups.

Using smaller-sized self-organizing maps, the M31 data are clustered into two major groups, which lead us to correlations that could not have otherwise been seen without clustering.
We found that six of the regions in M31 create a subgroup of data, that have relatively lower amount of Mid- and far-infrared emission.
PAH fluxes in these regions are highly anti-correlated with \halpha, \sii, \oiii~and IRAC 5.8~$\mu$m emission, stellar mass and RHI.
These anti-correlations are insignificant when data from the whole galaxy are considered.
We discussed possible reasons for these anti-correlations and argued that the most probable reason to PAHs anti-correlate with optical emission and RHI is having harder radiation.
Harder radiation makes the size of H {\sc II} regions bigger and the size of PDRs smaller, which cause more \halpha~emission and less PAHs, respectively.

PAHs in clustered data also show anti-correlation with stellar mass.
We discussed the possibility of that higher stellar mass might destroy PAHs and as a result we see anti-correlation between them.
However, we did not find any correlation between 8/24~$\mu$m emission, as a tracer of PAH abundance, and stellar mass.
Unlike previous studies, we also did not find any correlation between 8/250~$\mu$m emission and stellar mass. 
This results show that PAHs in the clustered regions in M31, do not heated by old stellar populations.

In the maps with larger sizes, networks can be used to illustrate the relative relations of the regions with one another.
We found the relative similarity between M31 regions, by varying the size of networks.
A one-dimensional network with 14 neurons was needed in order to
separate all 10 regions; we concluded that some of the region in M31 (e.g. regions 4 and 7) have very similar properties.

A two-dimensional networks provide us a more complete picture of the data.
We created various subsets from input data and generated different networks which helped us to show a relative relations between regions in M31 more clearly.
We showed that region 10 always become isolated in networks, except when we only used PAHs data to create a network.
In that network region 10 blends more with other regions and only shows differences with data from region 8.
These results suggest that the PAHs in all 10 regions have almost the same properties.

% We introduced self-organizing maps as a method that can be used to predict the amount of various quantities based on the relative positions of the regions in the networks.
% By converting the relative distance between regions to the probability distribution function, we could predict unobserved/unmeasured properties of the galaxies with 10 per cent uncertainty.
We also applied the SOMs from M31 data to M101 data and found regions with similar properties in both galaxies placed in close regions in the SOMs.
We show that using this ability of the SOM, we can predict properties of regions in other nearby galaxies very fast.
We should note that we created our networks using values for 26 quantities from 10 regions in M31, which does not consider as a "big sample".
Increasing dimension of the data and the number of regions would help to have more reliable networks.
In future we can apply these networks on data from other nearby galaxies (e.g. M33 and M83), and learn about the properties of regions on those galaxies.

 

\section*{ACKNOWLEDGMENTS}
The authors thank A. Tammour for her useful comments. 
S.R., P.B., and E.P. acknowledge research support from the Natural Sciences and Engineering Research Council of Canada. This research has made use of the NASA/IPAC Extragalactic Database (NED), which is operated by the Jet Propulsion Laboratory, California Institute of Technology, under contract with the National Aeronautics and Space Administration.
%----------------------------------------------------------------------------------------
%----------------------------------------------------------------------------------------
%----------------------------------------------------------------------------------------
%biblio
%----------------------------------------------------------------------------------------
%----------------------------------------------------------------------------------------
%----------------------------------------------------------------------------------------
\bibliographystyle{mn2e}
\bibliography{ref_mining.bib}
\import{../sections/}{app_2d.tex}

\end{document}
