\documentclass[useAMS,usenatbib]{mn2e}
\usepackage{amsmath}
\usepackage{hyperref}
\usepackage{graphicx}
\usepackage{natbib}
\bibliographystyle{mn2e}
\usepackage{times}
\usepackage{float}
\usepackage{caption}
\usepackage{subcaption}
\usepackage{multirow}
\usepackage{color,soul}
\usepackage{import}

\newcommand \kpc        {\,{\rm kpc}}
\newcommand \sigmagas    {$\Sigma_{\rm \bld {gas }} $\ }
\newcommand \sigmatotalgas {$\Sigma_{\rm \bld {total\, gas }} $\ }
\newcommand \eqsigmagas    {\Sigma_{\rm \bld {gas }}}
\newcommand \sigmasfr     {$\Sigma_{\rm \bld {SFR }} $\ }
\newcommand \eqsigmasfr     {\Sigma_{\rm \bld {SFR }}}
\newcommand \sigmastar    {$\Sigma_{\rm \bld {star }} $\ }
\newcommand \eqsigmastar    {\Sigma_{\rm \bld {star }}}
\newcommand \halpha    {H$\alpha $\ }
\newcommand \um    {$\mu$m\ }
\newcommand \mice {$\mu$m}
\newcommand \nprime {N$^\prime$}
\newcommand \boldit {\textbf{\textit{}}}
\newcommand \eqnprime {N^\prime}
\newcommand \Spitzer {{\it Spitzer }}
\newcommand \Galex {GALEX }
\newcommand \Herschel {{\it Herschel }}
\newcommand \aaj {A\&A}
\newcommand \aarv {A\&ARv}%: Astronomy and Astrophysics Review (the)
\newcommand \aas{A\&AS}%: Astronomy and Astrophysics Supplement Series
\newcommand \afz {Afz}%: Astrofizika
\newcommand \aj {AJ}%: Astronomical Journal (the)
\newcommand \apss {Ap\&SS}%: Astrophysics and Space Science
\newcommand \apj {ApJ}
\newcommand \apjs {ApJS}%: Astrophysical Journal Supplement Series (the)
\newcommand \araa {ARA\&A} %: Annual Review of Astronomy and Astrophysics
\newcommand \asp {ASP Conf. Ser.}%: Astronomy Society of the Pacific Conference Series
\newcommand \azh {Azh}%: Astronomicheskij Zhurnal
\newcommand \baas {BAAS}%: Bulletin of the American Astronomical Society
\newcommand \mem {Mem. RAS}%: Memoirs of the Royal Astronomical Society
\newcommand \mnassa {MNASSA}%: Monthly Notes of the Astronomical Society of Southern Africa
\newcommand \mnras {MNRAS} %: Monthly Notices of the Royal Astronomical Society
%\newcommand {Nature}%(do not abbreviate)
\newcommand \pasj {PASJ}%: Publications of the Astronomical Society of Japan
\newcommand \pasp {PASP}%: Publications of the Astronomical Society of the Pacific
\newcommand \qjras {QJRAS}%: Quarterly Journal of the Royal Astronomical Society
\newcommand \mex {Rev. Mex. Astron. Astrofis.}%: Revista Mexicana de Astronomia y Astrofisica
%\newcommand {Science }%}%(do not abbreviate)
\newcommand \sva {SvA}%: Soviet Astronomy
\newcommand \aap {APP} %:American Academy of Pediatrics
\newcommand \apjl {ApJL} %:The Astrophysical Journal Letters

\begin{document}
% TITLE

\title{Data Mining in Nearby Galaxies}
\author{rahmani.sahar }
\date{\today}
\author[S. Rahmani, et. al.]{S.~Rahmani$^{1}$\thanks{E-mail:
srahma49@uwo.ca}, H.~Teimoorinia$^{2}$, E.~Peeters$^{1}$, P.~Barmby$^{1}$\\
$^{1}$Department of Physics $\&$ Astronomy, Western University, London, ON N6A 3K7, Canada\\
$^{2}$Department of Physics $\&$ Astronomy, University of Victoria, Finnerty Road, Victoria, British Columbia, V8P 1A1, Canada}
\maketitle

%----------------------------------------------------------------------------------------
%----------------------------------------------------------------------------------------
%----------------------------------------------------------------------------------------
%abstract
%----------------------------------------------------------------------------------------
%----------------------------------------------------------------------------------------
%----------------------------------------------------------------------------------------

\begin{abstract} 
Here is the abstract.
%This is just first abstract for this paper!
%M31 was targeted by many probes, and there are variety of data available for this particular galaxy.
%M31 was mapped from X-ray to 21 cm emission. Also, spectroscopy data of different regions of M31 is available in various bands. 
%The vast availability of data for this galaxy makes it a suitable target for data mining and finding new patterns in this galaxy.
%In this project we are studying 10 regions in M31 from the bulge of the galaxy to the star forming ring. Using self organizing map (SOM) method, we classified these regions in different groups, based on total star formation rate, total stellar mass, total infrared emission and 3.6 \mice, 4.5 \mice, 5.8 \mice, 8 \mice, 24 \mice, 70 \um and 160 \mice, 2.6 mm, and 21 cm luminosity from the photometry data of the regions, and flux of 5.7 \mice, 6.2 \mice, 7.4 \mice, 7.6 \mice, 7.9\mice, 8.3 \mice, 8.6 \mice, 10.7 \mice, 11.23 \mice, 12.00 \mice, 12.62 \mice, 12.69 \mice, 14.0 \mice, 16.45 \mice, 17.04 \mice, PAH lines, [ArIII], [SIII], [SIV], [NeII], and [NeIII] as well as metallicity of these regions obtained from the spectroscopy data. 
%We divided these regions from 2 to 6 different classes, and for each class we are going to interpret results and check whether these classifications have a physical meaning or they are just statistically relate to each other. 
%As a result, we can predict some properties of regions in other galaxies by having only few data.

\end{abstract}
\begin{keywords} 
galaxies: individual: M31, galaxies: spiral, galaxies: star formation, galaxies: stellar content, galaxies: ISM, stars: formation, ISM: clouds, methods: observational, methods: statistical, data mining, methods:data analysis, techniques: image processing 
\end{keywords}
%----------------------------------------------------------------------------------------
%----------------------------------------------------------------------------------------
%----------------------------------------------------------------------------------------
%Intro
%----------------------------------------------------------------------------------------
%----------------------------------------------------------------------------------------
%----------------------------------------------------------------------------------------
\import{../sections/}{intro0.0.tex}
%----------------------------------------------------------------------------------------
%----------------------------------------------------------------------------------------
%----------------------------------------------------------------------------------------
%DATA
%----------------------------------------------------------------------------------------
%----------------------------------------------------------------------------------------
%----------------------------------------------------------------------------------------

\import{../sections/}{data0.0.tex}

%----------------------------------------------------------------------------------------
%----------------------------------------------------------------------------------------
%----------------------------------------------------------------------------------------
%Method
%----------------------------------------------------------------------------------------
%----------------------------------------------------------------------------------------
%----------------------------------------------------------------------------------------

\import{../sections/}{method0.0.tex}

%----------------------------------------------------------------------------------------
%----------------------------------------------------------------------------------------
%----------------------------------------------------------------------------------------
%Results
%----------------------------------------------------------------------------------------
%----------------------------------------------------------------------------------------
%----------------------------------------------------------------------------------------
\import{../sections/}{results0.0.tex}
%----------------------------------------------------------------------------------------
%----------------------------------------------------------------------------------------
%----------------------------------------------------------------------------------------
%Discussion
%----------------------------------------------------------------------------------------
%----------------------------------------------------------------------------------------
%----------------------------------------------------------------------------------------

\import{../sections/}{diss0.0.tex}

%----------------------------------------------------------------------------------------
%----------------------------------------------------------------------------------------
%----------------------------------------------------------------------------------------
%Summery
%----------------------------------------------------------------------------------------
%----------------------------------------------------------------------------------------
%----------------------------------------------------------------------------------------
\section{SUMMARY}

\section*{ACKNOWLEDGMENTS}
S.R and P.B acknowledge research support from the Natural Sciences and Engineering Research Council of Canada and from the Academic Development Fund of the University of Western Ontario.
%----------------------------------------------------------------------------------------
%----------------------------------------------------------------------------------------
%----------------------------------------------------------------------------------------
%biblio
%----------------------------------------------------------------------------------------
%----------------------------------------------------------------------------------------
%----------------------------------------------------------------------------------------
\bibliographystyle{mnras}
\bibliography{ref_mining}


\end{document}
