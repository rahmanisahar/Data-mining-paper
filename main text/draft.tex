\documentclass[useAMS,usenatbib]{mn2e}
\usepackage{amsmath}
\usepackage{hyperref}
\usepackage{graphicx}
\usepackage{natbib}
\usepackage{times}
\usepackage{float}
%\usepackage{caption}
\usepackage{subcaption}
\usepackage{multirow}
\usepackage{color,soul}
\usepackage{import}
\usepackage{ae,aecompl}
\usepackage{amssymb}	% Extra maths symbols
\usepackage{multicol}        % Multi-column entries in tables
\usepackage{bm}		% Bold maths symbols, including upright Greek
\usepackage{pdflscape}	% Landscape pages
%\usepackage{booktabs,fixltx2e}
\usepackage[flushleft]{threeparttable}
%\usepackage{mwe}    % loads »blindtext« and »graphicx«
\usepackage{slantsc}
\usepackage[T1]{fontenc}   


\newcommand \kpc        {\,{\rm kpc}}
\newcommand \sigmagas    {$\Sigma_{\mathrm {gas }} $\ }
\newcommand \sigmatotalgas {$\Sigma_{\mathrm {total\, gas }} $\ }
\newcommand \eqsigmagas    {\Sigma_{\mathrm {gas }}}
\newcommand \sigmasfr     {$\Sigma_{\mathrm {SFR }} $\ }
\newcommand \eqsigmasfr     {\Sigma_{\mathrm {SFR }}}
\newcommand \sigmastar    {$\Sigma_{\mathrm {star }} $\ }
\newcommand \eqsigmastar    {\Sigma_{\mathrm {star }}}
\newcommand \halpha    {H$\alpha $}
\newcommand \um    {$\mu$m\ }
\newcommand \mice {$\mu$m}
\newcommand \nprime {N$^\prime$}
\newcommand \boldit {\textbf{\textit{}}}
\newcommand \eqnprime {N^\prime}
\newcommand \Spitzer {{\it Spitzer }}
\newcommand \GALEX {GALEX }
\newcommand \Herschel {{\it Herschel }}
\newcommand \sii {[S~\textsc{ii}]} 
\newcommand \oiii {[O~\textsc{iii}]} 
\newcommand \hi {H~\textsc{i}\ }
\newcommand \hii {H~\textsc{ii}\ }

\newcommand \aaj {A\&A}
\newcommand \aarv {A\&ARv}%: Astronomy and Astrophysics Review (the)
\newcommand \aas{A\&AS}%: Astronomy and Astrophysics Supplement Series
\newcommand \afz {Afz}%: Astrofizika
\newcommand \aj {AJ}%: Astronomical Journal (the)
\newcommand \apss {Ap\&SS}%: Astrophysics and Space Science
\newcommand \apj {ApJ}
\newcommand \apjs {ApJS}%: Astrophysical Journal Supplement Series (the)
\newcommand \araa {ARA\&A} %: Annual Review of Astronomy and Astrophysics
\newcommand \asp {ASP Conf. Ser.}%: Astronomy Society of the Pacific Conference Series
\newcommand \azh {Azh}%: Astronomicheskij Zhurnal
\newcommand \baas {BAAS}%: Bulletin of the American Astronomical Society
\newcommand \mem {Mem. RAS}%: Memoirs of the Royal Astronomical Society
\newcommand \mnassa {MNASSA}%: Monthly Notes of the Astronomical Society of Southern Africa
\newcommand \mnras {MNRAS} %: Monthly Notices of the Royal Astronomical Society
\newcommand \pasj {PASJ}%: Publications of the Astronomical Society of Japan
\newcommand \pasp {PASP}%: Publications of the Astronomical Society of the Pacific
\newcommand \qjras {QJRAS}%: Quarterly Journal of the Royal Astronomical Society
\newcommand \mex {Rev. Mex. Astron. Astrofis.}%: Revista Mexicana de Astronomia y Astrofisica
\newcommand \sva {SvA}%: Soviet Astronomy
\newcommand \aap {APP} %:American Academy of Pediatrics
\newcommand \apjl {ApJL} %:The Astrophysical Journal Letters


\begin{document}
\title{Data Mining in Nearby Galaxies}
\author[S. Rahmani, et. al.]{S.~Rahmani$^{1, 2}$\thanks{E-mail:
srahma49@uwo.ca}, H.~Teimoorinia$^{3}$, E.~Peeters$^{1, 2, 4}$, P.~Barmby$^{1, 2}$\\
$^{1}$Department of Physics $\&$ Astronomy, Western University, London, ON N6A 3K7, Canada\\
$^{2}$Centre for Planetary Science \& Exploration, Western University, London, ON N6A 3K7, Canada\\
$^{3}$Department of Physics $\&$ Astronomy, University of Victoria, Finnerty Road, Victoria, British Columbia, V8P 1A1, Canada\\
$^{4}$SETI Institute, 189 Bernardo Avenue, Suite 100, Mountain View, CA 94043, USA}
\maketitle

%----------------------------------------------------------------------------------------
%----------------------------------------------------------------------------------------
%----------------------------------------------------------------------------------------
%abstract
%----------------------------------------------------------------------------------------
%----------------------------------------------------------------------------------------
%----------------------------------------------------------------------------------------

\begin{abstract} 
Galaxies are very complex systems; a complete understanding of them cannot be achieved only from studying linear or logarithmic correlations between their properties.
The vast availability of data for nearby galaxies makes them suitable targets for exploratory data analysis. 
In the past few decades, a number of statistical methods have been developed and advanced to study and visualize complex datasets.
We used one of these methods to study the properties of nearby galaxies and create a clearer picture of them.
%Spatially resolved maps provide us with a unique view of the inside of galaxies. %and help better understand their properties.
In this project, we utilized the Kohonen Self Organizing Map (SOM) method to study observations of the nearby galaxies M31 and M101. %SOM is an unsupervised neural network for mapping and visualizing a complex and nonlinear high dimension data while preserving topological features of the original data. 
%We studied 10 regions in M31 and 8 regions in M101. 
%These regions were chosen based on the availability of mid-infrared spectroscopy data.
%For each region in M31, we had 46 data obtained through photometry, spectroscopy, and derived quantities (i.e., star formation rate, stellar mass, gas mass, etc.). 
%Using results from a correlation coefficient matrix, we reduced the dimension of this dataset and then trained the SOM using the new subset of data. 
We created SOMs of various sizes and used them to extract information about the galaxies.
Using smaller-sized SOMs, we found that the data were clustered in 2 major groups; for each group, we found correlations that could not have otherwise been seen without clustering. 
In the maps with larger sizes, we created networks to illustrate the relative relations of the regions with one another. 
We then applied the SOMs that were generated from the M31 data to data from M101 and
found that regions with similar properties in both galaxies were placed in close regions in the SOMs.  
These results confirm that the generated SOMs can separate regions based on their physical properties and can be used to make predictions for other regions in nearby galaxies or other targets. 
\end{abstract}
\begin{keywords} %PB: I think MNRAS only lets you have up to 6 and they are supposed to be from here: http://journals.aas.org/authors/keywords2013.html
galaxies: individual: M31, M101, galaxies: spiral, galaxies: star formation, galaxies: stellar content, galaxies: ISM, stars: formation, ISM: clouds, methods: observational, methods: statistical, data mining, methods:data analysis, techniques: image processing 
\end{keywords}
%----------------------------------------------------------------------------------------
%----------------------------------------------------------------------------------------
%----------------------------------------------------------------------------------------
%Intro
%----------------------------------------------------------------------------------------
%----------------------------------------------------------------------------------------
%----------------------------------------------------------------------------------------
\import{../sections/}{intro0.03.tex}

%----------------------------------------------------------------------------------------
%----------------------------------------------------------------------------------------
%----------------------------------------------------------------------------------------
%Method
%----------------------------------------------------------------------------------------
%----------------------------------------------------------------------------------------
%----------------------------------------------------------------------------------------

\import{../sections/}{method0.03.tex}

%----------------------------------------------------------------------------------------
%----------------------------------------------------------------------------------------
%----------------------------------------------------------------------------------------
%DATA
%----------------------------------------------------------------------------------------
%----------------------------------------------------------------------------------------
%----------------------------------------------------------------------------------------

\import{../sections/}{data0.03.tex}

%----------------------------------------------------------------------------------------
%----------------------------------------------------------------------------------------
%----------------------------------------------------------------------------------------
%Results
%----------------------------------------------------------------------------------------
%----------------------------------------------------------------------------------------
%----------------------------------------------------------------------------------------
%\import{../sections/}{results0.03.tex}
\import{../sections/}{R10.03.tex}
\import{../sections/}{R20.03.tex}

%----------------------------------------------------------------------------------------
%----------------------------------------------------------------------------------------
%----------------------------------------------------------------------------------------
%Summery
%----------------------------------------------------------------------------------------
%----------------------------------------------------------------------------------------
%----------------------------------------------------------------------------------------
\section{Summary}
\label{sec: summary}

We present the results of studying spatially resolved regions in nearby galaxies using Kohonen self-organizing maps.
Self-organizing maps both cluster data and show relative distance between the clusters. 
We utilized the method to find hidden subgroups in the data as well as to find relations between those groups.

Using smaller-sized self-organizing maps, we clustered M31 data into two major groups, which led us to correlations that could not have otherwise been seen without clustering.
Six of the regions in M31 create a subgroup with relatively lower mid- and far-infrared emission.
PAH fluxes in these regions are highly anti-correlated with \halpha, \sii, \oiii~and IRAC 5.8~$\mu$m emission, stellar mass and radiation hardness index.
These anti-correlations are insignificant when the full dataset is considered.
The most probable reason for PAHs to anti-correlate with optical emission lines and RHI is a harder radiation field.
This makes the size of \hii~regions bigger and the size of photo-dissociation regions smaller, causing more \halpha~emission and less PAH emission, respectively.

PAHs in clustered data also show an anti-correlation with stellar mass.
We discussed the possibility that higher stellar mass might destroy PAHs;
however, we did not find any correlation between 8/24~$\mu$m emission, as a tracer of PAH abundance, and stellar mass.
Unlike previous studies, we also did not find any correlation between 8/250~$\mu$m emission and stellar mass. 
This result shows that PAHs in the clustered regions in M31 are not heated by the old stellar populations.
The reason behind the observed anti-correlation between PAH features and stellar mass needs further investigation, which requires mid-infrared PAH observations for more regions in M31. 

In self-organizing maps with larger sizes, networks can be used to illustrate the relations of the regions with one another.
We found the relative similarity between M31 regions by varying the size of the networks.
A one-dimensional network with 14 neurons was needed in order to
separate all 10 regions; some of the regions in M31 (e.g. regions 4 and 7) have very similar properties.
Two-dimensional networks provide us with a more complete picture of the data.
We created various subsets of the input data and generated different networks which helped to show relations between regions in M31 more clearly.
Region 10, which is located in the bulge of M31, always becomes isolated in the networks, except when we  used only PAH data to create a network.
In that network region 10 blends more with the other regions and is only differentiated from region 8, which is located inside the inner ring of M31.
These results suggest that the PAHs in all 10 regions have almost the same properties.

We introduced self-organizing maps as a method that can be used to predict various quantities based on the relative positions of the regions in the networks.
For regions that lack the observational data for some quantities (e.g. PAH emission), SOMs can be used to extrapolate the unobserved quantities.
Applying the SOMs from M31 to observations of M101, we found regions with similar properties in both galaxies placed in close regions in the SOMs.
Using this ability of the SOM, we can predict properties of regions in other nearby galaxies very quickly.

The networks we created used values for 26 quantities from 10 regions in M31, which is not a large sample.
Increasing the dimension of the data and the number of regions would yield more reliable networks.
Since there is very little spatially-resolved PAH spectroscopy for other galaxies from the \Spitzer Space Telescope, increasing the size of samples will require data from the James Webb Space Telescope.
The method described in this work can be used to select target regions for PAH observations with that facility. 

 

\section*{ACKNOWLEDGMENTS}
The authors thank Dr.s D.\ Stock and A.\ Tammour for useful comments. 
S.R., P.B., and E.P. acknowledge research support from the Natural Sciences and Engineering Research Council of Canada. This research has made use of the NASA/IPAC Extragalactic Database (NED), which is operated by the Jet Propulsion Laboratory, California Institute of Technology, under contract with the National Aeronautics and Space Administration.
%----------------------------------------------------------------------------------------
%----------------------------------------------------------------------------------------
%----------------------------------------------------------------------------------------
%biblio
%----------------------------------------------------------------------------------------
%----------------------------------------------------------------------------------------
%----------------------------------------------------------------------------------------
\bibliographystyle{mn2e}
\bibliography{ref_mining.bib}
\import{../sections/}{app_2d.tex}

\end{document}
