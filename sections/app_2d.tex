\newpage
\appendix
\section{2D SOMs of subsets}
\label{sec: app_2d_soms_SOMN}

<<<<<<< HEAD
        We removed IRAC 5.7 $\mu$m; \sii continuum and \oiii continuum; PAH8.3 $\mu$m, PAH12.0 $\mu$m, PAH17.0 $\mu$m, \oiii continuum, stellar mass and metallicity from subset 0, and generated subset samples 2 to 4, respectively (Tables~\ref{tab: subset2} to~\ref{tab: subset4}).
        In Fig.~\ref{fig: subset2}, which is the SOM results from data listed in Tab.~\ref{tab: subset2}, distance between positions of the region 7, and region 3 is much larger compare to the one in Fig.~\ref{fig: all_derived_ones}. 
        IRAC 5.7~$\mu$m flux in the region 7 and the region 3 is similar to each other. 
        Therefore, removing flux from IRAC 5.7 $\mu$m removed the one source of similarity in these two regions and they moved further from each other in the SOM.
        For subset sample 7 (Tab.~\ref{tab: subset7}) we removed \sii flux, SPIRE 250 and 500~$\mu$m and TIR emission from subset 2 (~\ref{tab: subset5}).
=======
      
        We removed IRAC 5.7 $\mu$m; \sii~continuum and \oiii~continuum; PAH8.3 $\mu$m, PAH12.0 $\mu$m, PAH17.0 $\mu$m, \oiii~continuum, stellar mass and metallicity from data from 10 regions in M31 that were described in Section~\ref{Sec: data_SOMN}, and generated subset samples 4 to 6, respectively (Tables~\ref{tab: subset2} to~\ref{tab: subset4}).
        In Fig.~\ref{fig: subset2}, which is the SOM results from data listed in Table~\ref{tab: subset2}, distance between positions of the region 7, and region 3 is much larger compare to the one in Fig.~\ref{fig: all_derived_ones}. 
        IRAC 5.7~$\mu$m flux in the region 7 and the region 3 is similar to each other. 
        Therefore, removing flux from IRAC 5.7~$\mu$m removed the one source of similarity in these two regions and they moved further from each other in the SOM.
        For subset sample 7 (Table~\ref{tab: subset7}) we removed \sii~flux, SPIRE 250 and 500~$\mu$m and TIR emission from subset 4 (~\ref{tab: subset5}).
>>>>>>> sharelatex-2016-07-26-1530
        The positions of the regions 2 and 9 are closer to each other, but the weight distance between them is much higher (the colours between them are became almost black). 
        In general by removing IRAC 5.7~$\mu$m flux data from the input data, the colours between regions become much more darker.
      
        In Fig.~\ref{fig: subset3}, the positions from the region 9 is much closer to the positions of the regions 4 and 7. 
<<<<<<< HEAD
        Since region 4 and 5 have a similar \oiii and \sii continuum fluxes, by removing these two values from the input data distance between regions 4 and 5 became much smaller.
        The positions of the regions 2 and 6 are closer together, but the colours between their neurons are darker.
        The wining neurons for regions 1 and 9 are moved further from each other, but the colours between them are slightly lighter. 
        moreover, similar to regions 4 and 5, \oiii and \sii continuum fluxes for regions 1 and 9, and regions 8 and 6 are similar to each other, and removing these two values from input data, reduces the similarity between these regions. Therefore, they moved further from each other. 
=======
        Since region 4 and 5 have a similar \oiii~and \sii~continuum fluxes, by removing these two values from the input data distance between regions 4 and 5 became much smaller.
        The positions of the regions 2 and 6 are closer together, but the colours between their neurons are darker.
        The wining neurons for regions 1 and 9 are moved further from each other, but the colours between them are slightly lighter. 
        moreover, similar to regions 4 and 5, \oiii~and \sii~continuum fluxes for regions 1 and 9, and regions 8 and 6 are similar to each other, and removing these two values from input data, reduces the similarity between these regions. Therefore, they moved further from each other. 
         \import{../sections/tables/}{subset2.tex}
        \import{../sections/image_texts/}{subset2.tex}
>>>>>>> sharelatex-2016-07-26-1530
        \import{../sections/tables/}{subset3.tex}
        \import{../sections/image_texts/}{subset3.tex}
        
        In the results from the subset 6 in Fig.~\ref{fig: subset4}, regions 8 and 6 moved closer to each other. 
        These two regions have distinct metallicity, removing metallicity from input data caused these two stay in a closer neurons. 
        Since regions 1 and 2 have similar metallicity and total PAHs, and their total stellar mass are comparable, removing these values from input data made the position of the regions 1 and 2 moved further from each other. 
        \import{../sections/tables/}{subset4.tex}
        \import{../sections/image_texts/}{subset4.tex}
        
        The positions of the region 8 and 5 are moved closer in Fig.~\ref{fig: subset7}.
        Both of these regions are around the inner ring inside the galaxy.
        Therefore, both of them have similar values in most of the input parameters.
<<<<<<< HEAD
        However, they have very different fluxes in IRAC 5.7 $\mu$m and \sii~continuum, which removing these two fluxes from the input data moved their position on the SOM closer.
        TIR luminosity and IRAC 5.7 $\mu$m flux for regions 3 and 5 are different, and removing them from the input data caused the distance between their position decreases. 
=======
        However, they have very different fluxes in IRAC 5.7~$\mu$m and \sii~emission, which removing these two fluxes from the input data moved their position on the SOM closer.
        TIR luminosity and IRAC 5.7~$\mu$m flux for regions 3 and 5 are different, and removing them from the input data caused the distance between their position decreases. 
>>>>>>> sharelatex-2016-07-26-1530
        In Fig.~\ref{fig: subset7} the relations between all the other regions are mostly unchanged.
        \import{../sections/tables/}{subset7.tex}
        \import{../sections/image_texts/}{subset7.tex}
        