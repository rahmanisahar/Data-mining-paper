 %----------------------------------------------------------------------------------------
%----------------------------------------------------------------------------------------
%----------------------------------------------------------------------------------------
%Result Part 2: 2D SOMs
%----------------------------------------------------------------------------------------
%----------------------------------------------------------------------------------------
%----------------------------------------------------------------------------------------
 \section{Two dimensional self-organizing maps}
 \label{sec: 2d_cluster}
    Although 1D networks are great to have a general idea about the data, neurons in 1D maps have maximum two neighbours, which causes some limitation in the results.
    In 2D networks, each neuron has two to six neighbours, which allowed them to to capture a whole picture of the complicated relations in the input data.
    Therefore, we created $10\times10$ 2D networks to study the data in details.
    As mentioned in previous sections, the size of SOMs are arbitrary and users must decide about it based on what their goals of using SOMs are.
    In this section we chose $10\times10$ size despite that we knew, from the size of our input data, that most of the neurons would be empty.
    Using 2D networks, we are mostly interested in the ability of the SOMs in showing underlying structure of the data rather than its clustering features.
    \import{../sections/image_texts/}{all.tex}
    Fig.~\ref{fig: all_derived_ones} shows the 2D SOM of M31 data.
    Similar to the Fig.~\ref{fig: M31_net_1by14}, all the 10 regions completely separated from each other in Fig.~\ref{fig: all_derived_ones}.
    
    Regions 4 and 7 are in the top left of Fig.~\ref{fig: all_derived_ones} with very bright colour between their neurons, which presents that these two regions are very similar.
    Considering the position of these two regions in M31 (Both of these regions are right on the edge of the star forming rings in the galaxy), similarity of these two regions were predictable.
    Position of region 3 on the SOM is close to the position of regions 4 and 7, but with darker colour between the nodes. 
    In Fig.~\ref{fig: regions in m31} it is clear that, this region in M31 is closer to regions 4 and 7 than any other region, but it is on the outer side of the star forming ring.
    Regions 5 and 6 are in their second neighbourhood on the SOM, with a medium gray colour between them.
    These two regions are around the inner ring of the galaxy.
    Regions 8 and 6 are both in the inner ring of M31, but region 6 is in the region with more star forming activities, which could be the main reason for their relative distance in the SOM in Fig.~\ref{fig: all_derived_ones}. 
    
    Regions 1 and 9 are close to each other, and in the same side of the star forming ring. 
    However, region 1 is in an area of the galaxy with lesser diffuse \halpha~emission than region 9, that might be the reason for their distances in SOM.
    Region 2 is more distant from other regions in the galaxy, and placed in the star forming ring.
    However, similar to regions 1 and 4, region 2 is located in an area of the galaxy with less diffused \halpha~emission.
    Therefore, place of region 2 in the SOM is more favoured towards regions 1 and 4. 
    Region 10 is placed in the bulge of the galaxy, and its position on the SOM is isolated from all the other regions by a strip of a dark colours. 
    
    In order to analyse effects of any input data on the final SOM, in the following we created SOMs from various subsets of data.
    We compare results of the subsets with one another and with SOM created from all data in Fig.~\ref{fig: all_derived_ones}.

    \subsection{Subsets}
    \label{sec: subsets}
            For analysing subsets, we creating SOMs by using only PAHs data, as well as using all data except PAHs ones (Fig.~\ref{fig: PAHS_or_not_PAHs}).
             \import{../sections/image_texts/}{PAHS_or_not_PAHs.tex}
            Comparing the SOM from all data in Fig.~\ref{fig: all_derived_ones} with the SOMs in Fig.~\ref{fig: PAHS_or_not_PAHs}, showed that the general position of regions in those networks are the same. 
            The region 10 is in one corner of the all three networks.
            However, in Fig.~\ref{fig: all_derived_ones} and Fig.~\ref{fig: wt_pahs}, region 10 is isolated from the other regions, while in Fig.~\ref{fig: only_pahs}, region 10 is much less isolated and only shows complete dissimilarity with region 8.
            Regions 9 and 10 are totally isolated in SOMs in Fig.~\ref{fig: wt_pahs}, but in Fig.~\ref{fig: only_pahs}, they are more similar to other regions.
            In both SOMs in Fig.~\ref{fig: PAHS_or_not_PAHs}, there are dissimilarities between regions in M31 but in Fig.~\ref{fig: wt_pahs}, the colours are much darker than the ones in Fig.~\ref{fig: only_pahs}.
            This indicates that there are more similarity between PAHs features from all 10 regions then the other input data.
            
            We increased the dimension of input data in Fig.~\ref{fig: only_pahs} (PAH only data), gradually, by adding data from other quantities of galaxies to the input data. 
            The order of adding data is the same as order of data in Fig.~\ref{fig: cor_all}, i.e. SOM in Fig~\ref{fig: inc_D_col3s}a was created from PAHs data and H$\alpha$ emission as input data, SOM in Fig~\ref{fig: inc_D_col3s}b created from PAHs, H$\alpha$ emission and [\sii] continuum as input data, and so on. 
            \import{../sections/image_texts/}{col3byall.tex}
            
            SOM algorithm assigns random weight to each network, that causes the overall position of regions in networks (even with the same input data) changes in each run of the algorithm.
            However, the position of each region regarding to other regions changes dramatically only because of the new sets of input data is given to the algorithm (assuming all the initial values for the run are intact).
            In all SOMs in Fig~\ref{fig: inc_D_col3s}, region 10 is located in the corner of the SOMs, but in  Figs~\ref{fig: inc_D_col3s}a,~\ref{fig: inc_D_col3s}d,~\ref{fig: inc_D_col3s}g,~\ref{fig: inc_D_col3s}k, and ~\ref{fig: inc_D_col3s}l, it is placed in the left side of the SOMs and in the others it is in the right side of the networks.
            In our discussion on differences between networks, we do not address the changes in the network due to initialization of the SOM algorithm and only consider the effect of changing the input data in the network.
            
            Comparing Figs.~\ref{fig: inc_D_col3s}a to ~\ref{fig: inc_D_col3s}o shows that adding \halpha~emission data to PAHs features causes regions 9 and 10 become isolated. 
            Increasing dimension of the input data makes region 10 more isolated.
            The relative position of regions 4 and 7 stays the same with increasing the input data. 
            Adding SPIRE 350, 500~$\mu$m emission and L$_{\rm dust}$ do not have any adequate effect on the networks.
            Stellar mass, total gas mass, dust mass and RHI data alternate the distances between neurons effectively, but it seems that adding SFR, L$_{\rm TIR}$ and metallicity data revoke those changes.
            
            We generated other subsets of the data based on the results in Figs~\ref{fig: inc_D_col3s} to study effects of the input data on SOMs, further.
            Subset 1, that are listed in Tab.~\ref{tab: subset1}, includes all the inputs data except data from stellar mass.
            Tab.~\ref{tab: subset5} is the list of data that were used in a subset 2, which includes all data except the fact that instead of the individual PAH fluxes we added them all together and have total PAH flux. 
            For subset 3 ( in Tab.~\ref{tab: subset6}), we removed SPIRE 250 and 500~$\mu$m from subset 2.
            Figs.~\ref{fig: subset1} --~\ref{fig: subset6} show SOMs that are created by data from subsets 1 to 3, respectively.
            The rest of the subsets are discussed in the Appendix.~\ref{sec: app_2d_soms_SOMN}.

            The results from Table~\ref{tab: subset1} is shown in Fig.~\ref{fig: subset1}. 
            In this SOM compare with the SOM from all the data in Fig.~\ref{fig: all_derived_ones}, regions 1 and 9 are closer to region 10. 
            Regions 1 and 9 are the ones with lowest stellar mass values, and region 10 has the highest stellar mass value among those 10 regions. 
            Since the differences in the amount of the stellar mass were one of the most distinct differences between these three regions, removing stellar mass from input data reduced the distance between these regions.
            Regions 5, 6 and 8 are in the same relative distance from each other as in Fig.~\ref{fig: all_derived_ones}, but they all are closer to the position of the region 10.
            Distance between regions 2 and 3 is reduced but in the meantime the colour between them became darker.

            \import{../sections/tables/}{subset1.tex}
            \import{../sections/image_texts/}{subset1.tex}

            Changing from separate values for each PAH features to a single value for the total PAHs caused small changes to the SOM map in Fig.~\ref{fig: subset5} compare to the one in Fig.~\ref{fig: all_derived_ones}. 
            Distance between the relative position of regions 6 and 8 increased significantly, while the position of region 2 moved closer to the positions of regions 4 and 7.
            The winner neurons for regions 3 and 5 moved towards each other, but the colours between them became much more darker. 

            \import{../sections/tables/}{subset5.tex}
            \import{../sections/image_texts/}{subset5.tex}

            Fig.~\ref{fig: subset6} shows the SOM generated from data that are listed in Tab.~\ref{tab: subset6}, which includes all the data from Tab.~\ref{tab: subset5} except SPIRE 250 and 500~$\mu$m emission.
            Although for most of the regions we see the changes in their positions, the colours between neurons changed, too. 
            %Therefore, we can conclude that these changes caused by the differences in the initial assigned weights.
            The distance between the positions of the regions 4 and 2 are increased.
            The same happened for distances between regions 7, 3 and 6.
            In both cases, SPIRE 250 and 500~$\mu$m emission of these regions are similar to one another, and removing these two parameters from input data moved the positions of the regions further from each other. 
            \import{../sections/tables/}{subset6.tex}
            \import{../sections/image_texts/}{subset6.tex}
            
            %Subsets, that some of them  are studied in this section, can be used to learn about the effect of each input on the map as well as they can be used to predict unobserved quantities in the galaxy.
            \subsubsection{Prediction observing data using M31}
        
                On SOMs we can see relations between the input data from each region regarding to the other regions.
                These relations are shown by colour in SOMs, when white is 100 per cent similarity and black is 0 per cent similarity.
                Therefore, we have the probability distribution of quantities of each values given data from other regions.
                We can use these probability distribution to estimate missing data for regions. 
                
                To demonstrate this conclusion, we assumed the stellar mass value for region 1 is unknown.
                Therefore, we need a SOM that are generated from all the inputs data for all regions except stellar mass (e.g subset 1 in Fig.~\ref{fig: subset1}).
                \import{../sections/image_texts/}{sim_subset1.tex}
                We found the shortest path between region 1 and the other regions and measure their relative similarity ($p_j$).
                Assuming the maximum similarity is 100 per cent between two neurons (white colour), we multiply the similarity values along the shortest path between two regions to find $p_j$ between region 1 and the other regions (Fig.~\ref{fig: sim_subset1}).
                In both Figs.~\ref{fig: subset1} and ~\ref{fig: sim_subset1} are clear that region 1 shows more similarity to regions 2 and 9 than the others and the most dissimilarity to region 8. 
                
                We measured probability of stellar mass given other quantities of the region 1 ($P(m_1\mid \forall_1)$) using Equation~\ref{equ: prob1}.
                \begin{equation}
                \label{equ: prob1}
                    P(m_1\mid \forall_1) = \sum_{j=2}^{10}p_j*m_j
                \end{equation}
                Where $m_1$ is the stellar mass of region 1 and $m_j$ is the stellar mass of region j (any other region).
                We estimated the stellar mass to be $\sim1377$~M$\odot$ which is 10 per cent less than the observed values.
                This type of the prediction from SOM networks can be used to predict observation values in observational proposals or can be used in pre-phase studies of the big missions such as the James Webb Space Telescope (JWST) and LLST.
                

    \subsection{Validating networks using M101 data}
    Another application of SOM for nearby galaxies is that to apply the trained networks on the data from other galaxies in order to prediction properties of the regions in them. %%eh
    As an example of this application, we generated an SOM using data from M31 that held in common with M101 data (Fig.~\ref{fig: subset9}a). 
    In this SOM, the same as the others, region 10 is separated from other regions in M31.
    Regions 7 and 4 are close to each other with very light colours between their neurons, and the same as the other SOMs regions 6, 8, and 5 are close to each other, but with medium dark colour between them.
    \import{../sections/image_texts/}{subset9.tex}
    
    We applied the SOM created using M31 data in Fig.~\ref{fig: subset9}a on M101 data (Fig.~\ref{fig: subset9}b).
    Region 6 in M31 and region 2 in M101 both occupied the same neuron in the network in Fig.~\ref{fig: subset9}, which immediately suggest that these two regions have similar properties. 
    Region 2 in M101 is a bright H {\sc II} region that is located at the end of the one of the M101 spiral arm (see Fig.~\ref{fig: regions in m101}). 
    Region 6 in M31 is located near the inner ring of the galaxy (see Fig.~\ref{fig: regions in m31}).
    Both regions have relatively low amount of PAHs emission and medium amount of the SFR regarding the other regions in the galaxy, which makes them very similar regions.
    
    Region 7 in M101 is located in the nucleus of the galaxy and have relatively higher quantities than the other regions.
    This region is located in the top right side of the network in Fig.~\ref{fig: subset9}, which is close to location of region 10 in the network.
    Since the bulge of a spiral galaxy has different environment from its nucleus, region 7 in M101 and region 10 in M31 are not occupied the same neuron in the SOM, and have a medium grey colour between their neurons.

    Region 1 in M101 is located near the nucleus of M101 (see Fig.~\ref{fig: regions in m101}), but has considerably lower values in all the quantities than the ones in region 7.
    The lower values for fluxes of the PAH features, and relatively moderate amount of the SFR caused this region places between region 4 and 5 of M31 in the network. 
    Fig.~\ref{fig: subset9}b shows that region 4 in M101 is separated itself from other regions and located between region 9 and 10 in M31.
    \cite{Gordon08} showed that this region is a diffuse nebula in M101, with high amount of fluxes of PAH features. 
    This region also shows a high amount of the SFR and the stellar mass, which describes why the location of region 4 from M101 in the SOM is close to regions 9 and 10 from M31.
    
    In this section we showed that we can use networks that are created by data from nearby galaxies, to study the properties of other galaxies in a fast way.
    We should note that we created our networks using values for 26 quantities from 10 regions in M31, which does not consider as a "big sample".
    Increasing dimension of the data and the number of regions would help to have more reliable networks.
    
    